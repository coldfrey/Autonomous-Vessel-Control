% Author: Joshua Carey
% Description: Introduction chapter.
\let\textcircled=\pgftextcircled
\chapter{Introduction and Motivation}\label{chap:intro}

\initial{M}aritime travel has been a cornerstone of human civilization, facilitating the exchange of goods, ideas, and cultures across vast expanses of water. The annals of history are full with instances of seafaring civilizations harnessing the power of wind to propel their vessels across the oceans. It is posited that ancient Neanderthals embarked on maritime voyages in the southern Ionian Islands between 110 to 35ka BP$~$\cite{Ferentinos2012}. The quintessence of maritime travel has predominantly been wind-powered sails, which remained unchallenged until the industrial revolution ushered in the era of fuel-powered engines.

The art and science of sailing have evolved significantly over millennia, from rudimentary rafts and canoes to sophisticated sailing ships with complex rigging systems. Ancient civilizations, including the Egyptians, Phoenicians, and Polynesians, made remarkable advancements in sailing technology, enabling them to explore and trade over larger swathes of the ocean$~$\cite{casson1995ships}. The medieval period saw the advent of the compass and the astrolabe, which further facilitated maritime navigation and exploration. The Age of Discovery, epitomized by the voyages of Columbus, Vasco da Gama, and Magellan, was propelled by advancements in sailing technology, which enabled transoceanic voyages and the establishment of maritime empires.

The industrial revolution in the 18th and 19th centuries marked a significant turning point in maritime propulsion. The invention of the steam engine heralded the decline of wind-powered sailing and the rise of fuel-powered propulsion systems. Steam-powered ships and later, diesel-powered ships, offered greater reliability, speed, and capacity compared to their wind-powered predecessors, thus becoming the preferred mode of maritime transportation$~$\cite{gardiner1993advent}. The transition to fuel-powered engines also mirrored the broader industrial and technological advancements of the era, which prioritized speed, efficiency and profit over traditional methods.

\section{A Renewed Interest in Wind Propulsion}
However, the environmental costs of fuel-powered maritime transportation have become increasingly apparent in the modern era. The shipping industry is a notable contributor to global carbon emissions, and the negative effects of pollution on marine ecosystems around the world are well-documented$~$\cite{corbett2007mortality}. These challenges have rekindled interest in wind propulsion as a sustainable alternative, prompting a re-examination of the principles that guided ancient and medieval sailors. The modern iteration of wind propulsion seeks to amalgamate the age-old wisdom of harnessing wind power with contemporary technological advancements to create eco-friendly and efficient maritime transportation systems.

Contemporary wind propulsion technologies like Flettner rotors$~$\cite{vahs2019retrofitting}, wing sails, and kite systems are being revisited to mitigate the environmental impact of maritime travel. Among these, kite-powered vessel technology stands out due to its potential for higher efficiency and ease of retrofitting. Kites offer two main advantages over traditional sails: they can move relative to the vessel, generating their own apparent wind and can be flown at higher altitudes, accessing different wind systems.

The relative movement of kites generates apparent wind, allowing for maximum potential force even when the vessel is stationary. This enhanced apparent wind results in a larger force compared to a sail of equivalent area. Flying kites at higher altitudes taps into different wind systems and currents to those at sea level, making wind a potentially more reliable energy source for propulsion.

However, the effective operation of kite-powered vessels requires precise control, which is skill-intensive. To leverage the full benefits of kites as a scalable propulsion method, implementing autonomous control is crucial. 

Reinforcement Learning (RL), a subset of artificial intelligence, presents a compelling avenue for optimizing the autonomous control of kite-powered vessels. RL, which is based on learning through interaction with an environment, offers a potential for developing advanced control systems and strategies that could greatly improve the effectiveness of kite-powered vessels. 


% The objective of this paper is to delineate a novel Reinforcement Learning approach aimed at optimizing the autonomous control of kite-powered vessels. By leveraging the prowess of RL, we endeavor to address the challenges inherent in autonomous sail control, thereby contributing to the overarching goal of sustainable and efficient maritime transportation. The ensuing discourse will elucidate the theoretical underpinnings of our approach, the experimental setup, and the empirical findings, underscoring the potential benefits in terms of reducing operational costs and mitigating environmental impact.


\section{Aims and Objectives}

The overarching aim of this research is to develop a robust Reinforcement Learning (RL) algorithm capable of autonomously controlling a kite-powered vessel. This objective stems from the need to advance sustainable maritime travel technologies and reduce the environmental impact of current propulsion systems.

To achieve this primary aim, the objectives have been structured as follows:

\subsection*{Objective 1: Simulation Environment Development}
\begin{itemize}
    \item To design and implement a virtual marine environment that accurately emulates real-world maritime conditions.
    \item To construct a realistic model of a boat that exhibits appropriate physical movements in response to environmental forces such as wind and water currents.
\end{itemize}

\subsection*{Objective 2: Kite Propulsion Modeling}
\begin{itemize}
    \item To create a physics-based model of a kite within the simulation that reflects authentic aerodynamic behaviors.
    \item To integrate kite control mechanics into the agent’s available actions, allowing for simulated propulsion through wind energy harnessing.
\end{itemize}

\subsection*{Objective 3: Reinforcement Learning Framework Establishment}
\begin{itemize}
    \item To formulate a set of observations, actions, and rewards that encapsulate the dynamics of autonomous kite-boat navigation.
    \item To deploy the Proximal Policy Optimization algorithm, leveraging its actor-critic method for effective policy learning.
\end{itemize}

\subsection*{Objective 4: Autonomous Agent Development}
\begin{itemize}
    \item To develop an RL agent capable of learning basic control and maneuvers, starting with simple navigating towards a target and maintaining a constant course.
    \item To refine the agent's capability to adaptively control the kite's position and angle to optimize propulsion for speed while navigating towards a target in any direction.
\end{itemize}

\subsection*{Objective 5: Efficacy and Optimization Testing}
\begin{itemize}
    \item To rigorously evaluate the trained agent’s performance in simulating autonomous navigation in various environmental scenarios within the developed Unity environment.
    \item To utilize High-Performance Computing (HPC) resources for scaling up simulations and optimizing the training process.
\end{itemize}

\subsection*{Objective 6: Real-World Applicability Assessment}
\begin{itemize}
    \item To extrapolate simulation findings to assess real-world applicability and propose a framework for practical deployment of RL in kite-powered vessels.
    \item To provide recommendations for further research and development based on empirical results obtained from the simulation studies.
\end{itemize}

These objectives pave the path towards achieving the central goal, ensuring each phase of development builds upon the last. By concluding this investigation with assessments geared towards real-world implementation, it is anticipated that the contributions made will significantly impact sustainable maritime transportation solutions.



% @article{Ferentinos2012,
%     author = {Ferentinos, G. and Gkioni, M. and Geraga, M. and Papatheodorou, G.},
%     title = {Early seafaring activity in the southern Ionian Islands, Mediterranean Sea},
%     journal = {Journal of Archaeological Science},
%     volume = {39},
%     number = {7},
%     pages = {2167-2176},
%     year = {2012},
%     doi = {10.1016/J.JAS.2012.01.032},
%     url = {https://dx.doi.org/10.1016/J.JAS.2012.01.032}
% }
% @book{Balard2017,
%     editor = {Balard, Michel and Buchet, Christian},
%     title = {The Sea in History - The Medieval World},
%     publisher = {Boydell \& Brewer, Boydell Press},
%     year = {2017},
%     edition = {NED - New edition},
%     pages = {1086},
%     url = {https://www.jstor.org/stable/10.7722/j.ctt1kgqt6m}
% }
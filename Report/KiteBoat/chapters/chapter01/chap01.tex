% Author: Joshua Carey
% Description: Introduction chapter.
\let\textcircled=\pgftextcircled
\chapter{Introduction and Motivation}\label{chap:intro}

\initial{M}aritime travel has been a cornerstone of human civilization, facilitating the exchange of goods, ideas, and cultures across vast expanses of water. The annals of history are full with instances of seafaring civilizations harnessing the power of wind to propel their vessels across the oceans. It is posited that ancient Neanderthals embarked on maritime voyages in the southern Ionian Islands between 110 to 35ka BP$~$\cite{Ferentinos2012}. The quintessence of maritime travel has predominantly been wind-powered sails, which remained unchallenged until the industrial revolution ushered in the era of fuel-powered engines.

The art and science of sailing have evolved significantly over millennia, from rudimentary rafts and canoes to sophisticated sailing ships with complex rigging systems. Ancient civilizations, including the Egyptians, Phoenicians, and Polynesians, made remarkable advancements in sailing technology, enabling them to explore and trade over larger swathes of the ocean$~$\cite{casson1995ships}. The medieval period saw the advent of the compass and the astrolabe, which further facilitated maritime navigation and exploration. The Age of Discovery, epitomized by the voyages of Columbus, Vasco da Gama, and Magellan, was propelled by advancements in sailing technology, which enabled transoceanic voyages and the establishment of maritime empires.

The industrial revolution in the 18th and 19th centuries marked a significant turning point in maritime propulsion. The invention of the steam engine heralded the decline of wind-powered sailing and the rise of fuel-powered propulsion systems. Steam-powered ships and later, diesel-powered ships, offered greater reliability, speed, and capacity compared to their wind-powered predecessors, thus becoming the preferred mode of maritime transportation$~$\cite{gardiner1993advent}. The transition to fuel-powered engines also mirrored the broader industrial and technological advancements of the era, which prioritized speed, efficiency and profit over traditional methods.

\section{A Renewed Interest in Wind Propulsion}
The environmental costs of fuel-powered maritime transportation have become increasingly apparent in the modern era. The shipping industry is a notable contributor to global carbon emissions, and the negative effects of pollution on marine ecosystems around the world are well-documented$~$\cite{corbett2007mortality}. These challenges have rekindled interest in wind propulsion as a sustainable alternative, prompting a re-examination of the principles that guided ancient and medieval sailors. The modern iteration of wind propulsion seeks to amalgamate the age-old wisdom of harnessing wind power with contemporary technological advancements to create eco-friendly and efficient maritime transportation systems.

Contemporary wind propulsion technologies like Flettner rotors$~$\cite{vahs2019retrofitting}, wing sails, and kite systems are being revisited to mitigate the environmental impact of maritime travel. Among these, kite-powered vessel technology stands out due to its potential for higher efficiency and ease of retrofitting. Kites offer two main advantages over traditional sails: they can move relative to the vessel, generating their own apparent wind and can be flown at higher altitudes, accessing different wind systems.

The relative movement of kites generates apparent wind, allowing for maximum potential force even when the vessel is stationary. This enhanced apparent wind results in a larger force compared to a sail of equivalent area. Flying kites at higher altitudes taps into different wind systems and currents to those at sea level, making wind a potentially more reliable energy source for propulsion.

However, the effective operation of kite-powered vessels requires precise control, which is skill-intensive. To leverage the full benefits of kites as a scalable propulsion method, implementing autonomous control is crucial. 

Reinforcement Learning (RL), a subset of artificial intelligence, presents a compelling avenue for optimizing the autonomous control of kite-powered vessels. RL, which is based on learning through interaction with an environment, offers a potential for developing advanced control systems and strategies that could greatly improve the effectiveness of kite-powered vessels. 


% The objective of this paper is to delineate a novel Reinforcement Learning approach aimed at optimizing the autonomous control of kite-powered vessels. By leveraging the prowess of RL, we endeavor to address the challenges inherent in autonomous sail control, thereby contributing to the overarching goal of sustainable and efficient maritime transportation. The ensuing discourse will elucidate the theoretical underpinnings of our approach, the experimental setup, and the empirical findings, underscoring the potential benefits in terms of reducing operational costs and mitigating environmental impact.


\section{Project Definition}
This project aims to develope a simulation environment that can be used to train an RL agent to autonomously control a kite-powered vessel. The development of a simulation environment is crucial, attempting to train an agent to fly a kite let alone navigate a boat in the real world would not be feasible; the simulation environment will allow for the agent to learn in a safe and controlled fashion, not to mention the iteration time and cost will be significantly reduced. However simulation creates its own issues. Integration into the real world becomes an significant factor, how realistic is good enough? What features of the real world are the most important to implement? How do we know if the agent will be able to perform in the real world? These are all questions that will be considered throughout the project. 

The kite-powered vessel in question is displayed in figure XYZ, and is comprised of a boat hull and LEI (leading edge inflatable) kite, connected by 4 control lines, this configuration is further explained in section$~$\ref{sec:env}. 

Autonomous control is a broad term, in this case it means enabling self-governance of the vessel control functions with no human input. For the purpose of this project the agent will be trained to navigate the kiteboat towards a target location (waypoint). The better the agent performs this task, the harder the problem will become with waypoints moving to different locations. This process if taken to its extreme could represent full autonomous control, with the entire navigation and control down to the agent. `Control', what does control mean in this scenario. The agent will have a number of actions, these will represent steering the boats rudder and flying the kite. A single agent will be trained to control both the kite and the boat, this makes the learning problem more complicated but was attempted as a novel approach to create a system that had full control of the vessel. In this vein success is two-fold, the agent must be able to fly the kite reliably while continuously steering towards the next waypoint.

For any machine learning endeavor, especially one with such intricate physical dynamics, the choice of simulation environment is paramount. Not only does it provide the playground for our AI agent to learn and make mistakes safely, but it also serves as a litmus test for the robustness and realism of the designed model.

Given the myriad of choices available, the Unity game engine emerged as the most suitable platform. Its native support for machine learning applications through the MLAgents toolkit was unmatched. Beyond its reputation in gaming, the inherent support for mesh bodies, colliders, and a variety of joints made it an attractive option for simulating the kite-boat system, which comprised a complex dance of forces, and counterforces. Unity is recognized for its potent physics engine, however for this project the use of Unity's physics engine will be kept to a minimum, with an emphasis on implementing as much of the simulation from scratch as possible. Unity will be used primarily for its visual capabilities and the ability to be used as the gym environment for machine learning simulations. 

Central to this simulation is the depiction of water, the medium in which the boat will navigate. Here, the Unity HDRP Water System 16.0.3$~$\cite{UnityHDRPWaterSystem}, bundled with Unity 2023.2.0b9, provides a realistic representation of water with its undulating waves, refractions, and reflections. 

The alternative option to using Unity's water system was to model an entire particle fluid simulation, this would have had its advantages, however it would have been a lot more computationally expensive and would have taken a lot longer to implement. As this project was primarily focused on training a RL algorithm for controlling a kiteboat it was decided that the Unity water system would be sufficient for this project.  






\section{Aims and Objectives}

The overarching aim of this research is to develope a system for controlling kite-powered vessels using Reinforcement Learning (RL) techniques. This objective stems from the need to advance sustainable maritime travel technologies and reduce the environmental impact of current propulsion systems. 

To achieve this primary aim, the objectives have been structured as follows:

\subsection*{Objective 1: Simulation Environment Development}
\begin{itemize}
    \item To design and implement a virtual marine environment that accurately emulates real-world maritime conditions.
    \item To construct a realistic model of a boat that exhibits appropriate physical movements in response to environmental forces such as wind and water currents.
    \newline\textbf{Outcome Goal:} To have a physics-based boat able to be controlled and driven around a scene by a human player.
\end{itemize}

\subsection*{Objective 2: Kite Propulsion Modeling}
\begin{itemize}
    \item To create a physics-based model of a kite within the simulation that reflects authentic aerodynamic behaviors and integrate it onto the boat model.
    \item To integrate kite control mechanics into an agent’s available action space.
    \newline\textbf{Outcome Goal:} To have a physics-based kiteboat able to be controlled and driven around a scene by a human player, using an agent's heuristic controls.
\end{itemize}

\subsection*{Objective 3: Reinforcement Learning Framework Establishment}
\begin{itemize}
    \item To formulate a set of observations, actions, and rewards that encapsulate the dynamics of autonomous kite-boat control and navigation.
    \item To deploy the Proximal Policy Optimization algorithm, leveraging its actor-critic method for effective policy learning.
    \newline\textbf{Outcome Goal:} To have an agent begin training using PPO to learn to control the kiteboat.
\end{itemize}

\subsection*{Objective 4: Autonomous Agent Development}
\begin{itemize}
    \item To develop an RL agent capable of learning basic control and maneuvers, starting with simple navigating towards a target and maintaining a constant course.
    \item To refine the agent's capability to adaptively control the kite's position and angle to optimize propulsion for speed while navigating towards a target.
    \newline\textbf{Outcome Goal 1:} To have an agent that can navigate towards a target in a straight line.
    \newline\textbf{Outcome Goal 2:} To have an agent that can navigate towards a target in any direction, including using maneuvers to take the optimal path.
\end{itemize}

\subsection*{Objective 5: Efficacy and Optimization Testing}
\begin{itemize}
    \item To utilize High-Performance Computing (HPC) resources for scaling up simulations and optimizing the training process.
    \item To rigorously evaluate the trained agent’s performance in simulating autonomous navigation in various environmental scenarios.
    \newline\textbf{Outcome Goal 1:} To train an agent using the HPC resources.
    \newline\textbf{Outcome Goal 2:} To have an agent that can navigate towards a target in a straight line under various environmental conditions, including wind and waves.
\end{itemize}

\subsection*{Objective 6: Real-World Applicability Assessment}
\begin{itemize}
    \item To extrapolate simulation findings to assess real-world applicability and propose a  practical deployment of RL in kite-powered vessels.
    \item To provide recommendations for further research and development based on empirical results obtained from the simulation studies.
\end{itemize}

These objectives pave the path towards achieving the central goal, ensuring each phase of development builds upon the last. By concluding this investigation it is anticipated that the contributions made could impact sustainable maritime transportation solutions.



% @article{Ferentinos2012,
%     author = {Ferentinos, G. and Gkioni, M. and Geraga, M. and Papatheodorou, G.},
%     title = {Early seafaring activity in the southern Ionian Islands, Mediterranean Sea},
%     journal = {Journal of Archaeological Science},
%     volume = {39},
%     number = {7},
%     pages = {2167-2176},
%     year = {2012},
%     doi = {10.1016/J.JAS.2012.01.032},
%     url = {https://dx.doi.org/10.1016/J.JAS.2012.01.032}
% }
% @book{Balard2017,
%     editor = {Balard, Michel and Buchet, Christian},
%     title = {The Sea in History - The Medieval World},
%     publisher = {Boydell \& Brewer, Boydell Press},
%     year = {2017},
%     edition = {NED - New edition},
%     pages = {1086},
%     url = {https://www.jstor.org/stable/10.7722/j.ctt1kgqt6m}
% }
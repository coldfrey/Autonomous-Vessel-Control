

\let\textcircled=\pgftextcircled\chapter{Results and Validation}\label{chap:results}

Introduction
Briefly summarize the goals of the RL experiments.
Outline the structure of the results chapter.

\section{Final Experimental Setup}
The training scene used for the final training runs can be found in:
\newline
\texttt{Assets/Scenes/kiteboat\_training}. The final experimental setup was a combination of the best performing elements from the previous experiments. The model was configured with the config file shown in section$~$\ref{config} of the appendix. 


Outline the training procedure, including any pre-training steps.

\section{Training Results}
Training Results
Present the learning curves for the agent, showing reward over time.
Include a table or graph of the agent’s performance metrics at various checkpoints.
Discuss any unexpected behaviors or anomalies observed during training.

\subsection{Hyperparameter Tuning}\label{sec:hyperparameter_tuning}

Show the graphs with the results of the grid search and explain why the config set was chosen
\subsection{Final Training}

Present the graphs of the final training runs and discuss the results

\section{Agent Performance}

\subsection*{Difficulties Encountered}
There were several difficulties encountered when trying to get the agent to learn anything let alone the combination of directionally sailing a kiteboat. One of the most common local maxima that the agent fell into was for the agent to steer on `hard lock' with the rudder at \textit{$\tilde{} ~70-90^{\circ}$}, shown in figure$~$\ref{hard_lock} where the target can be seen as the green area in the distance. This behavior allowed it to learn to fly the kite very reliably with the boat in a more consistent and stable position. These episodes provided false positives in the training data because as soon as the agent started to explore the rudder space more it was not able to fly the kite. To try and combat this behavior a large negative reward was added for aggressive steering as shown in table$~$\ref{rewards}. This went some way to discouraging this behavior but it was still observed in some of the later training runs. After this rudder reward was added it was observed that the agent took almost 5 times as long to learn to fly the kite with some reliability. 

\begin{figure}[!htb]
    \centering
    \includegraphics[width=0.8\textwidth]{Images/hard_lock.png}
    \caption{The agent steering on hard lock}\label{hard_lock}
\end{figure}


Evaluation of Agent Performance
Explain the methods used to evaluate the trained agent (e.g., test episodes in varied conditions).
Present the results of these evaluations in a systematic manner (e.g., tables, graphs).
Discuss the agent’s ability to generalize from its training to new scenarios.

Comparison with Baselines
If there are baseline models or industry standards, compare your results against these.
Use statistical methods to determine the significance of the results.

Ablation Studies
Discuss any ablation studies conducted (if any) to understand the contribution of different components of the system.
Present the impact of removing/modifying certain parts of the agent or the environment.

Visualizations and Case Studies
Provide visual representations of agent behavior, such as plots of trajectories or state visitation frequencies.
Include screenshots or video stills from Unity to illustrate successful and unsuccessful episodes.

Summary
Sum up the main findings from the results.
Discuss any limitations of these results or the experimental setup.


\section{Evaluation Introduction}
Evaluation
Introduction
Reflect on the purpose of the evaluation.

Methodology for Evaluation
Explain the metrics used to evaluate the agent’s performance.
Describe the process of collecting evaluation data.

Performance Analysis
Analyze the agent’s performance in depth, possibly breaking down by different types of tasks or challenges it faced.
Use statistical methods to discuss the performance variability.

Robustness and Generalization
Evaluate the agent's robustness to changes in the environment (e.g., different wind conditions, system perturbations).
Discuss the agent's ability to generalize from its training environment to similar but unseen environments.

Comparison to Human Performance
If relevant, compare the agent’s performance to that of a human completing the same task.
Discuss any qualitative differences in approaches to the task between the agent and humans.

Discussion of the Evaluation Results
Interpret the results in the context of the project's objectives.
Discuss the strengths and weaknesses of the agent as revealed by the evaluation.

Implications for Future Work
Suggest how the results might inform future projects or the development of RL agents for similar tasks.
Discuss any additional experiments or data that would be valuable.

Conclusion
Summarize the main takeaways from the evaluation.
Discuss the conclusions that can be drawn about the RL agent's performance and the potential for real-world application.

\subsection{Technical Limitations}\label{sec:limitations}
There were several limitations that affected the quality of the training and thus the trained model. First and foremost was compute, as expected when conducting any machine learning training, the more compute available the better. The local machine used for training was a 16-core i9 with 32GB of RAM and a T2000 nvidia graphics, and took approximately 2 hours per million steps completed. This was not a viable option for training the agent to a high level of performance, and so the training was attempted to the university HPC. The HPC has 525 Lenovo nx360 m5 nodes each with two 14 core 2.4 GHz CPUs, and 32 GPU nodes with two cards each. At face value this looks wonderful and training should be a breeze, however the problem is two fold. First Unity does not support native multi threading, due to the complex nature of its physics engine, so it runs on a single CPU unless manually specified. Manual threading was possible but only for separate tasks that could be called independently of the model, such as the collision detection algorithm. As expected this severely limits the speed of training. The second more major problem was the wait time of the HPC, the longer a job was scheduled the longer it took for the resources to be allocated, for a job that was scheduled for 48 hours on a single node it could take up to 5 days for the job to begin. This was a major problem as it meant that the training could not be conducted in a timely manner, and iteration was very slow. The limited GPU's had the same problem, with only 32 available, and the fact that the HPC was shared with the entire university, it was very difficult to get access to the resources. For this reason the vast majority of training was conducted on the local machine, which had the advantage of also being able to see the training in real time, albeit slowly.   
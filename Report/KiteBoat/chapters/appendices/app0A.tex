%
% file: localoperator.tex
% author: Victor Brena
% description: Briefly describes properties of the local operator.
%

\chapter{Appendix A}\label{app:app01}

% \section{Unity Setup}\label{sec:unity_setup}

\section{Final Configuration}\label{config}
\begin{adjustbox}{width=\textwidth/2}
\begin{lstlisting}
    behaviors:
  BoatAgent:
    trainer_type: ppo
    hyperparameters:
      batch_size: 256
      buffer_size: 4096
      learning_rate: 3.0e-4
      beta: 5.0e-4
      epsilon: 0.3
      lambd: 0.99
      num_epoch: 4
      learning_rate_schedule: constant
    network_settings:
      normalize: false
      hidden_units: 256
      num_layers: 6
    reward_signals:
      extrinsic:
        gamma: 0.99
        strength: 1.0
      curiosity:
        strength: 0.1
        gamma: 0.99
        learning_rate: 3.0e-4
    max_steps: 30000000
    time_horizon: 64
    summary_freq: 10000
    checkpoint_interval: 50000
    keep_checkpoints: 25
\end{lstlisting}
\end{adjustbox}

\section{Scripts}

% \subsection{KiteBoat Agent}\label{sec:kiteboat_agent}
% \lstset{
%   breaklines=true,   % sets automatic line breaking
%   breakatwhitespace=false,    % sets if automatic breaks should only happen at whitespace
% }
% \lstset{
%   basicstyle=\ttfamily\small,
%   breaklines=true
% }
  
% \begin{lstlisting}
%   using System.Collections;
%   using System.Collections.Generic;
%   using UnityEngine;
%   using Unity.MLAgents;
%   using Unity.MLAgents.Sensors;
%   using Unity.MLAgents.Actuators;
%   using PathCreation.Examples;
%   using System.IO;
  
%   public class KiteBoatAgent : Agent
%   {
%       public Rudder rudder;
  
%       public KiteSim kiteSim;
  
%       public Rigidbody kiteRigidbody;
  
%       public Rigidbody boatRigidbody;
  
%       public WaypointManager waypointManager;
  
%       private GJKCollisionDetection gjkCollisionDetection = new GJKCollisionDetection();
  
%       private const int RudderActionSize = 3; // left, none, right for rudder
%       private const int KiteControlActionSize = 3; // left, none, right for kite control
  
%       private Vector3 previousPosition; // To store the previous position
  
%       private float boatForwardSpeed;
  
%       private float currentDistanceToWaypoint;
%       private float previousDistanceToWaypoint;
  
%       private float checkInterval = 5f;
%       private Coroutine checkPositionRoutine;
  
%       public UnityEngine.UI.Text rewardText;
  
%       public UnityEngine.UI.Text actionText;
  
%       public UnityEngine.UI.Text completedWaypointsText;
%       public UnityEngine.UI.Text lessonText;
  
%       private Vector3 startPosition;
%       private Quaternion startRotation;
  
%       private Vector3 startKitePosition;
  
%       private Quaternion startKiteRotation;
  
%       private bool hitFinish = false;
  
%       private bool isFirstEpisode = true;
  
%       public int lessonNumber = 0;
      
%       public float goneTheDistance = 0;
  
%       public float lvl0Success = 5;
%       private bool hitLessonTarget = false;
  
%       private const float KiteStableHeight = 10f; // Example stable height for the kite
%       private const float KiteHeightRewardMultiplier = 0.1f;
  
%       private int[] lastActions = new int[2];
%       private float currentRudderAngle = 0f;
%       private float lastRudderAngle = 0f;
  
%       private StreamWriter writer;
  
%       private int episodeCount = 0;
  
  
%       public override void OnEpisodeBegin()
%       {
%           // jointSim.ResetKite();
%           if (isFirstEpisode)
%           {
%               isFirstEpisode = false;
%               startPosition = transform.position;
%               // startRotation = transform.rotation;
%               startKitePosition = kiteRigidbody.transform.position;
%               startKiteRotation = kiteRigidbody.transform.rotation;
  
%               string folderPath = "ML_PPO/HeuristicLogs/";
%               string fullPath = Path.Combine(Application.dataPath, folderPath);
%               if (!Directory.Exists(fullPath))
%               {
%                   Directory.CreateDirectory(fullPath);
%               }
  
%               // var behaviorParameters = GetComponent<BehaviorParameters>();
%               // // Determine the player type based on the BehaviorType
%               // string playerTypePrefix = (behaviorParameters.BehaviorType == BehaviorType.HeuristicOnly) ? "human_" : "model_";
%               // string fileName = playerTypePrefix + "Log_" + System.DateTime.Now.ToString("yyyy-MM-dd_HH-mm-ss") + ".csv";
%               // string filePath = Path.Combine(fullPath, fileName);
  
%               // Now the directory exists, proceed to create the StreamWriter
%               string filePath = Path.Combine(fullPath, "HeuristicLog_" + System.DateTime.Now.ToString("yyyy-MM-dd_HH-mm-ss") + ".csv");
%               writer = new StreamWriter(filePath, true);
%               writer.AutoFlush = true;
%               writer.WriteLine("Episode, Step, Time, Cumulative_Reward, Lesson, Waypoints_Reached, Distance, Boat_Speed, Rudder_Angle, Kite_Height");
%               print("Created new file at: " + filePath);
  
%           }
%           if (lessonNumber != 0 && waypointManager.GetWaypoint() == null)
%           {
%               waypointManager.CreateWaypoint();
%           }
%           if (lessonNumber == 3) checkInterval = 10f;
%           if (lessonNumber == 4) checkInterval = 20f;
%           // ResetEnvironment();
%           StartCoroutine(ResetEnvironment());
%           previousPosition = transform.position;
%           if (checkPositionRoutine != null)
%           {
%               StopCoroutine(checkPositionRoutine);
%           }
%           checkPositionRoutine = StartCoroutine(CheckPosition());
%           // StartCoroutine(WaitAndRequestDecision());
%           SetReward(0f);
%           Debug.Log("OnEpisodeBegin, post ResetEnvironment");
%           episodeCount ++;
%           RequestDecision();
%       }
%       private IEnumerator ResetEnvironment() 
%       {
  
%           yield return null;
          
%           kiteRigidbody.velocity = Vector3.zero;
%           kiteRigidbody.angularVelocity = Vector3.zero;
%           boatRigidbody.velocity = Vector3.zero;
%           boatRigidbody.angularVelocity = Vector3.zero;
%           boatRigidbody.transform.GetChild(0).GetChild(0).GetComponent<Rigidbody>().velocity = Vector3.zero;
%           boatRigidbody.transform.GetChild(0).GetChild(0).GetComponent<Rigidbody>().angularVelocity = Vector3.zero;
          
%           rudder.angle = 0f;
%           rudder.SetRudderTargetAngle(0f);
%           kiteSim.StopSteering();
  
%           startRotation = Quaternion.Euler(0, Random.Range(90, 270), 0);
%           transform.SetPositionAndRotation(startPosition + new Vector3(0, 2, 0), startRotation);
%           kiteRigidbody.transform.SetPositionAndRotation(startKitePosition, startKiteRotation);
  
%           if (hitFinish)
%           {
%               waypointManager.NewWaypoint(lessonNumber);
%               hitFinish = false;
              
%           }
%           kiteRigidbody.velocity = Vector3.zero;
%           kiteRigidbody.angularVelocity = Vector3.zero;
%           boatRigidbody.velocity = Vector3.zero;
%           boatRigidbody.angularVelocity = Vector3.zero;
%           boatRigidbody.transform.GetChild(0).GetChild(0).GetComponent<Rigidbody>().velocity = Vector3.zero;
%           boatRigidbody.transform.GetChild(0).GetChild(0).GetComponent<Rigidbody>().angularVelocity = Vector3.zero;
  
%           // goneTheDistance = 0;
%           hitLessonTarget = false;
%           if (lessonNumber != 0 && waypointManager.GetWaypoint() == null)
%           {
%               waypointManager.CreateWaypoint();
%           }
  
%           yield return new WaitForSeconds(1.0f);
%       }
      
  
%       public void FixedUpdate()
%       {
%           // try catch edge cases
%           if (GetCumulativeReward() < -100.0f || rudder.GetBoatForwardSpeed() > 100.0f || transform.position.y > 10.0f)
%           {
%               print("GetCumulativeReward: " + GetCumulativeReward() + ", rudder.GetBoatForwardSpeed(): " + rudder.GetBoatForwardSpeed() + ", transform.position.y: " + transform.position.y);
%               CompleteEpisode();
%           }
  
%           boatForwardSpeed = rudder.GetBoatForwardSpeed();
  
%           // reward: greater +ve velocity
%           if (boatForwardSpeed > 1f) {
%               AddReward(boatForwardSpeed * 0.005f);
%               if (boatForwardSpeed > 4f)
%               {
%                   AddReward(boatForwardSpeed * 0.01f);
%               }
%           }
%           else if (boatForwardSpeed < 0.1f)
%           {
%               AddReward(-0.001f);
%           }
%           // reward if the boat is moving towards the waypoint
%           if (lessonNumber != 0 && waypointManager.GetWaypoint() != null)
%           {
%               currentDistanceToWaypoint = Vector3.Distance(transform.position, waypointManager.GetWaypoint().transform.position);
%               previousDistanceToWaypoint = Vector3.Distance(previousPosition, waypointManager.GetWaypoint().transform.position);        
%               if (currentDistanceToWaypoint < previousDistanceToWaypoint)
%               {
%                   // AddReward(0.001f);
%                   // make the reward inversely proportional to the distance
%                   AddReward(10f / currentDistanceToWaypoint);
%               }
%               else if (currentDistanceToWaypoint > previousDistanceToWaypoint)
%               {
%                   AddReward(-0.5f * (currentDistanceToWaypoint- previousDistanceToWaypoint));
%               }
%               previousPosition = transform.position;
  
%               if (gjkCollisionDetection.GJK(gameObject.GetComponent<MeshFilter>(), transform.position, waypointManager.GetWaypoint().GetComponent<MeshFilter>(), waypointManager.GetWaypoint().transform.position))
%               {
%                   Debug.Log("Finish");
%                   hitFinish = true;
%                   AddReward(200.0f);
%                   // CompleteEpisode();
%                   waypointManager.NewWaypoint(lessonNumber);
%               }
  
%               currentRudderAngle = rudder.angle;
%               if (Mathf.Abs(currentRudderAngle) == Mathf.Abs(lastRudderAngle) && Mathf.Abs(currentRudderAngle) > 45.0f)
%               {
%                   AddReward(-0.001f * (Mathf.Abs(currentRudderAngle)));
%               }
%               lastRudderAngle = currentRudderAngle;
              
%           }
  
%           AddReward(-0.001f);
%           CheckLessonSuccess();
%           completedWaypointsText.text = "Completed Waypoints: " + waypointManager.GetCompletedWaypoints().ToString();
%           if (lessonNumber == 0)
%           {
%               lessonText.text = "Lesson: 0, " + goneTheDistance.ToString() + "/" + lvl0Success.ToString() + ", " + (500f - Vector3.Distance(transform.position, startPosition)).ToString("0.0") + "m to go";
%           }
%           else 
%           {
%               lessonText.text = "Lesson: " + lessonNumber.ToString();
%           }
%           if (Academy.Instance.StepCount % 100 == 0)
%           {
%               print("Logging data step: " + Academy.Instance.StepCount);
%               LogData();
%           }
%       }
  
%       private void CheckLessonSuccess()
%       {
%           // to graduate from lesson 0 to 1, the boat must travel 500m from the start position 5 times
%           if (lessonNumber == 0 && Vector3.Distance(transform.position, startPosition) > 500f)
%           {
%               if (hitLessonTarget) return;
%               goneTheDistance ++;
%               hitLessonTarget = true;
%               if (goneTheDistance >= lvl0Success)
%               {
%                   lessonNumber ++;
%                   waypointManager.CreateWaypoint();
%                   goneTheDistance = 0;
%               }
%           }
%           // to graduate from lesson 1 to 2, the boat must complete 10 waypoints in the 45 degree window
%           if (lessonNumber == 1 && waypointManager.GetCompletedWaypoints() > 10f)
%           {
%               lessonNumber ++;
%           }
%           if (lessonNumber == 2 && waypointManager.GetCompletedWaypoints() > 50f)
%           {
%               lessonNumber ++;
%           }
%           if (lessonNumber == 3 && waypointManager.GetCompletedWaypoints() > 80f)
%           {
%               lessonNumber ++;
%           }
%           if (lessonNumber == 4 && waypointManager.GetCompletedWaypoints() > 150f)
%           {
%               lessonNumber ++;
%           }
  
%       }
  
%       private IEnumerator CheckPosition()
%       {
%           while (true)
%           {
%               yield return new WaitForSeconds(checkInterval);
%               if (lessonNumber != 0)
%               {
%                   float startDistanceToWaypoint = Vector3.Distance(startPosition, waypointManager.GetWaypoint().transform.position);
%                   // If the boat if further away now than it started end episode
%                   if (currentDistanceToWaypoint >= startDistanceToWaypoint + 10.0f)
%                   {               
%                       print("currentDistanceToWaypoint: " + currentDistanceToWaypoint + ", started at position: " + startDistanceToWaypoint);
%                       AddReward(-10.0f);
%                       CompleteEpisode();
%                   }
%               }
%           }
%       }
  
%       private Vector3 PolarAngles(Vector3 kitePosition, Vector3 wind)
%       {
%           Vector3 basePosition = boatRigidbody.position + new Vector3(0, 1f, 0);
%           // Debug.DrawLine(basePosition, basePosition + kitePosition, Color.green, 1.0f);
%           // Debug.Log("kitePosition: " + kitePosition);
%           // wind projected on floor
%           Vector3 windOnFloor = new Vector3(wind.x, 0, wind.z);
%           // kite position projected on floor
%           Vector3 kitePositionOnFloor = Vector3.ProjectOnPlane(kitePosition, Vector3.up);
%           // angle between wind and kite
%           // Debug.DrawLine(basePosition, basePosition + windOnFloor, Color.red, 1.0f);
%           // Debug.DrawLine(basePosition, basePosition + kitePositionOnFloor, Color.blue, 1.0f);
  
%           float angleHorizontal = Vector3.Angle(windOnFloor, kitePositionOnFloor);
%           bool angleHorizontalSign = Vector3.Cross(windOnFloor, kitePositionOnFloor).y < 0;
%           if (angleHorizontalSign)
%           {
%               angleHorizontal = -angleHorizontal;
%           }
%           // wind projected on vertical plane as defined by direction of kitePositionOnFloor and up.
%           // Vector3 windOnVerticalPlane = Vector3.ProjectOnPlane(wind, Vector3.Cross(kitePositionOnFloor, Vector3.up));
%           // Debug.DrawLine(basePosition, basePosition + windOnVerticalPlane, Color.yellow, 1.0f);
%           // angle between windOnVerticalPlane and kitePosition
%           float angleVertical = Vector3.Angle(kitePositionOnFloor, kitePosition);
%           // kite transform.forward projected on plane defined by kitePositionOnFloor
%           Vector3 kiteForwardOnPlane = Vector3.ProjectOnPlane(kiteRigidbody.transform.forward, kitePositionOnFloor);
%           // Debug.DrawLine(basePosition, basePosition + kiteForwardOnPlane, Color.magenta, 1.0f);
%           // angle between kiteForwardOnPlane and up
%           float angleKiteUp = Vector3.Angle(kiteForwardOnPlane, Vector3.up);
  
%           return new Vector3(angleHorizontal, angleVertical, angleKiteUp);
  
%       }
  
%       public override void CollectObservations(VectorSensor sensor)
%       {
%           Vector3 boatPosition = transform.position;
%           Vector3 boatForward = transform.forward;
%           Vector3 wind = kiteSim.windAtKite;
%           float distanceToWaypoint;
%           if (lessonNumber == 0)
%           {
%               distanceToWaypoint = 0;
%           }
%           else 
%           {
%               Vector3 waypointPosition = waypointManager.GetWaypoint().transform.position;
%               distanceToWaypoint = Vector3.Distance(boatPosition, waypointPosition);
%           }
  
%           // Distance to waypoint
%           sensor.AddObservation(distanceToWaypoint);
%           // Boat speed
%           sensor.AddObservation(boatForwardSpeed);
%           // Wind
%           sensor.AddObservation(wind);
%           // kite polar angles
%           sensor.AddObservation(PolarAngles(kiteRigidbody.transform.localPosition, wind));
  
%           // Calculate the relative wind direction in the horizontal plane
%           Vector3 windDirectionOnHorizontalPlane = Vector3.ProjectOnPlane(wind, Vector3.up).normalized;
%           Vector3 boatForwardOnHorizontalPlane = Vector3.ProjectOnPlane(boatForward, Vector3.up).normalized;
  
%           // Calculate the angle using atan2 for a smooth transition
%           float angleToWind = Mathf.Atan2(
%               Vector3.Dot(Vector3.Cross(boatForwardOnHorizontalPlane, windDirectionOnHorizontalPlane), Vector3.up),
%               Vector3.Dot(boatForwardOnHorizontalPlane, windDirectionOnHorizontalPlane)
%           );
  
%           // Normalize the angle to be between 0 and 1
%           float normalizedAngleToWind = (angleToWind + Mathf.PI) / (2 * Mathf.PI);
  
%           sensor.AddObservation(normalizedAngleToWind);
  
%           // Additional observations for kite control
%           Vector3 kiteVelocityRelative = kiteRigidbody.velocity - boatRigidbody.velocity;
%           sensor.AddObservation(kiteVelocityRelative); // Relative velocity of kite to boat
  
%           float normalizedKiteHeight = Mathf.Clamp01((kiteRigidbody.transform.localPosition.y - 1f) / (KiteStableHeight - 1f));
%           sensor.AddObservation(normalizedKiteHeight); // Kite height relative to water
  
  
%           float normalizedRudderAngle = rudder.angle / 90.0f;
%           sensor.AddObservation(normalizedRudderAngle);
  
%           // Kite position relative to boat, normalised relative position rotated to boat view and distance
%           // Vector3 kiteRelativePosition = kiteRigidbody.transform.position - transform.position;
%           // Vector3 kiteRelativePositionNormalised = kiteRelativePosition.normalized;
%           // sensor.AddObservation(Quaternion.Inverse(transform.rotation) * kiteRelativePositionNormalised);
%           // sensor.AddObservation(kiteRelativePosition.magnitude);
%           // print("angle of boat to wind: " + angleToWind * Mathf.Rad2Deg);
  
%       }
  
%       public override void OnActionReceived(ActionBuffers actions)
%       {
  
%           int rudderAction = actions.DiscreteActions[0];
%           int kiteControlAction = actions.DiscreteActions[1];
  
%           actionText.text = "Rudder: " + rudderAction + ", Kite Control: " + kiteControlAction;
  
%           // discrete actions
%           // Rudder control (left, right, none)
%           if (rudderAction == 1)
%           {
%               rudder.RudderLeft();
%           }
%           else if (rudderAction == 2)
%           {
%               rudder.RudderRight();
%           }
  
%           // Kite control (left, right, none)
%           if (kiteControlAction == 0)
%           {
%               kiteSim.SteerLeft();
%           }
%           else if (kiteControlAction == 1)
%           {
%               kiteSim.StopSteering();
%           }
%           else if (kiteControlAction == 2)
%           {
%               kiteSim.SteerRight();
%           }
  
%           if (lastActions[0] != rudderAction || lastActions[1] != kiteControlAction)
%           {
%               AddReward(-0.1f);
%           }
%           lastActions[0] = rudderAction;
%           lastActions[1] = kiteControlAction;
  
%           // if the kite is less that 1m above the water, end the episode
%           if (kiteRigidbody.transform.localPosition.y < 1.0f)
%           {
%               AddReward(-10f);
%               CompleteEpisode();
%           }
  
%           if (PolarAngles(kiteRigidbody.transform.localPosition, kiteSim.windAtKite).y >15)
%           {
%               AddReward(0.01f);
%               if (PolarAngles(kiteRigidbody.transform.localPosition, kiteSim.windAtKite).y > 45.0f)
%               {
%                   AddReward(0.05f);
%               }
%           }
  
%           // negativly reward if the rudder is greater than 45 degrees by increasing the reward the further away from 45 degrees
%           if (Mathf.Abs(rudder.angle) > 50.0f)
%           {
%               AddReward(-0.01f * (Mathf.Abs(rudder.angle) - 50.0f));
%           }
  
%           // float heightDifference = Mathf.Abs(kiteRigidbody.transform.localPosition.y - KiteStableHeight);
%           // if (heightDifference < 2f) // Kite is within the desired height range
%           // {
%           //     AddReward((2f - heightDifference) * KiteHeightRewardMultiplier);
%           // }
%           // else
%           // {
%           //     AddReward(-heightDifference * KiteHeightRewardMultiplier);
%           // }
  
%           rewardText.text = "Reward: " + GetCumulativeReward().ToString("0.0000");
  
%           RequestDecision();
%       }
  
%       public override void Heuristic(in ActionBuffers actionsOut)
%       {
%           // Implementation of heuristic for manual control
%           // It's useful for testing the agent with keyboard input
%           var discreteActionsOut = actionsOut.DiscreteActions;
%           discreteActionsOut.Clear();
  
%           if (Input.GetKey(KeyCode.A))
%           {
%               discreteActionsOut[0] = 1; // Rudder left
%           }
%           else if (Input.GetKey(KeyCode.D))
%           {
%               discreteActionsOut[0] = 2; // Rudder right
%           }
  
%           if (Input.GetKey(KeyCode.LeftArrow))
%           {
%               discreteActionsOut[1] = 0; // Kite control left
%           }
%           else if (Input.GetKey(KeyCode.RightArrow))
%           {
%               discreteActionsOut[1] = 2; // Kite control right
%           }
%           else
%           {
%               discreteActionsOut[1] = 1; // Kite control none
%           }
%       }
  
  
%       private void LogData()
%       {
%           // writer.WriteLine("Episode, Step, Reward, Lesson, Waypoints_Reached, Distance, Episode_Length, Boat_Speed, Rudder_Angle, Kite_Height");
  
%           try
%           {
%               writer.WriteLine(episodeCount + "," + Academy.Instance.StepCount + "," + Time.time + "," + GetCumulativeReward() + "," + lessonNumber + "," + waypointManager.GetCompletedWaypoints() + "," + Vector3.Distance(transform.position, startPosition) + "," + boatForwardSpeed + "," + rudder.angle + "," + kiteRigidbody.transform.localPosition.y);
  
%               writer.Flush();
%           }
%           catch (System.Exception e)
%           {
%               print("Error writing to file: " + e);
%           }
%       }
  
%       void OnDestroy()
%       {
%           // Close the writer when the agent is destroyed
%           if (writer != null)
%           {
%               writer.Flush();
%               writer.Close();
%               writer = null;
%           }
%       }
%       public void CompleteEpisode()
%       {
%           // ResetEnvironment();
%           StartCoroutine(ResetEnvironment());
%           StopCoroutine(checkPositionRoutine);
%           // StartCoroutine(WaitAndRequestDecision());
%           // ResetEnvironment();
%           EndEpisode();
%       }
  
%   }
% \end{lstlisting}

% \subsection{Buoyancy}\label{sec:buoyancy}

% \begin{lstlisting}
% using System.Collections.Generic;
% using UnityEngine;
% using UnityEngine.Rendering.HighDefinition;

% public class Buoy : MonoBehaviour
% {
%     public WaterSurface seaSurface;
%     WaveSearchParameters waveParameters = new WaveSearchParameters();
%     WaveSearchResult waveResult = new WaveSearchResult();
%     public float buoyancyDensity = 500;
%     public int divisionsPerDimension = 2;
%     public bool hasConcaveShape = false;
%     public int maxVoxels = 16;

%     public float RESISTANCE = 0.3f;
%     public float LIQUID_DENSITY = 1000;

%     private float halfVoxelHeight;
%     private Vector3 archimedesForceLocal;
%     private List<Vector3> voxelPoints;
%     private bool hasMeshCollider;

%     private void Start()
%     {
%         var initialRotation = transform.rotation;
%         var initialPosition = transform.position;
%         transform.rotation = Quaternion.identity;
%         transform.position = Vector3.zero;

%         if (GetComponent<Collider>() == null)
%         {
%             gameObject.AddComponent<MeshCollider>();
%             Debug.LogWarning($"[SeaFloat.cs] Object \"{name}\" had no collider. MeshCollider has been added.");
%         }
%         hasMeshCollider = GetComponent<MeshCollider>() != null;

%         var colliderBounds = GetComponent<Collider>().bounds;
%         halfVoxelHeight = Mathf.Min(colliderBounds.size.x, colliderBounds.size.y, colliderBounds.size.z) / (2 * divisionsPerDimension);

%         if (GetComponent<Rigidbody>() == null)
%         {
%             gameObject.AddComponent<Rigidbody>();
%             Debug.LogWarning($"[SeaFloat.cs] Object \"{name}\" had no Rigidbody. Rigidbody has been added.");
%         }
%         GetComponent<Rigidbody>().centerOfMass = new Vector3(0, -10f, -1) + transform.InverseTransformPoint(colliderBounds.center);

%         voxelPoints = DivideIntoVoxels(hasMeshCollider && hasConcaveShape);

%         transform.rotation = initialRotation;
%         transform.position = initialPosition;

%         float objectVolume = GetComponent<Rigidbody>().mass / buoyancyDensity;
%         CombineVoxels(voxelPoints, maxVoxels);

%         float buoyantForceMagnitude = LIQUID_DENSITY * Mathf.Abs(Physics.gravity.y) * objectVolume;
%         archimedesForceLocal = new Vector3(0, buoyantForceMagnitude, 0) / voxelPoints.Count;
%     }

%     private List<Vector3> DivideIntoVoxels(bool concave)
%     {
%         var points = new List<Vector3>(divisionsPerDimension * divisionsPerDimension * divisionsPerDimension);
%         var objectBounds = GetComponent<Collider>().bounds;
%         if (concave)
%         {
%             var meshCollider = GetComponent<MeshCollider>();
%             bool originalConvex = meshCollider.convex;
%             meshCollider.convex = false;

%             SliceConcaveShape(meshCollider, objectBounds, points);
%             meshCollider.convex = originalConvex;
%         }
%         else
%         {
%             SliceConvexShape(objectBounds, points);
%         }

%         return points;
%     }

%     private void SliceConcaveShape(MeshCollider meshCollider, Bounds bounds, List<Vector3> points)
%     {
%         for (int x = 0; x < divisionsPerDimension; x++)
%         {
%             for (int y = 0; y < divisionsPerDimension; y++)
%             {
%                 for (int z = 0; z < divisionsPerDimension; z++)
%                 {
%                     Vector3 point = CalculatePoint(bounds, x, y, z);
%                     if (IsPointInside(meshCollider, point))
%                     {
%                         points.Add(point);
%                     }
%                 }
%             }
%         }

%         if (points.Count == 0)
%         {
%             points.Add(bounds.center);
%         }
%     }

%     private void SliceConvexShape(Bounds bounds, List<Vector3> points)
%     {
%         for (int x = 0; x < divisionsPerDimension; x++)
%         {
%             for (int y = 0; y < divisionsPerDimension; y++)
%             {
%                 for (int z = 0; z < divisionsPerDimension; z++)
%                 {
%                     Vector3 point = CalculatePoint(bounds, x, y, z);
%                     points.Add(point);
%                 }
%             }
%         }
%     }

%     private Vector3 CalculatePoint(Bounds bounds, int x, int y, int z)
%     {
%         float posX = bounds.min.x + bounds.size.x / divisionsPerDimension * (0.5f + x);
%         float posY = bounds.min.y + bounds.size.y / divisionsPerDimension * (0.5f + y);
%         float posZ = bounds.min.z + bounds.size.z / divisionsPerDimension * (0.5f + z);
%         return transform.InverseTransformPoint(new Vector3(posX, posY, posZ));
%     }

%     private static bool IsPointInside(Collider collider, Vector3 point)
%     {
%         Vector3[] directions = { Vector3.up, Vector3.down, Vector3.left, Vector3.right, Vector3.forward, Vector3.back };
%         foreach (var direction in directions)
%         {
%             if (!collider.Raycast(new Ray(point - direction * 1000, direction), out _, 1000f))
%             {
%                 return false;
%             }
%         }
%         return true;
%     }

%     private static void FindNearestPoints(IList<Vector3> points, out int indexA, out int indexB)
%     {
%         float minDistance = float.MaxValue;
%         indexA = 0;
%         indexB = 1;

%         for (int i = 0; i < points.Count - 1; i++)
%         {
%             for (int j = i + 1; j < points.Count; j++)
%             {
%                 float distance = Vector3.Distance(points[i], points[j]);
%                 if (distance < minDistance)
%                 {
%                     minDistance = distance;
%                     indexA = i;
%                     indexB = j;
%                 }
%             }
%         }
%     }

%     private static void CombineVoxels(IList<Vector3> points, int targetCount)
%     {
%         if (points.Count <= 2 || targetCount < 2)
%         {
%             return;
%         }

%         while (points.Count > targetCount)
%         {
%             FindNearestPoints(points, out int first, out int second);

%             Vector3 combinedPoint = (points[first] + points[second]) * 0.5f;
%             points.RemoveAt(second); // Remove second point first as its index is higher
%             points.RemoveAt(first);
%             points.Add(combinedPoint);
%         }
%     }

%     private float DetermineWaterLevel(float xCoord, float zCoord)
%     {
%         if (seaSurface == null)
%         {
%             Debug.Log("seaSurface is null");
%             return 0.0f;
%         }

%         if (seaSurface.ProjectPointOnWaterSurface(waveParameters, out waveResult))
%         {
%             return waveResult.projectedPositionWS.y;
%         }
%         else
%         {
%             return 0.0f;
%         }
%     }

%     private void FixedUpdate()
%     {
%         foreach (var voxel in voxelPoints)
%         {
%             var worldPoint = transform.TransformPoint(voxel);
%             float seaLevel = DetermineWaterLevel(worldPoint.x, worldPoint.z);

%             if (worldPoint.y - halfVoxelHeight < seaLevel)
%             {
%                 float displacement = (seaLevel - worldPoint.y) / (2 * halfVoxelHeight) + 0.5f;
%                 displacement = Mathf.Clamp(displacement, 0f, 1f);

%                 var velocity = GetComponent<Rigidbody>().GetPointVelocity(worldPoint);
%                 var dampingForce = -velocity * RESISTANCE * GetComponent<Rigidbody>().mass;
%                 var buoyancyForce = dampingForce + Mathf.Sqrt(displacement) * archimedesForceLocal;

%                 GetComponent<Rigidbody>().AddForceAtPosition(buoyancyForce, worldPoint);
%             }
%         }
%     }
% }

% \end{lstlisting}

% \subsection{Rudder}\label{sec:rudder}

% \begin{lstlisting}
%   using UnityEngine;

% public class Rudder : MonoBehaviour
% {
%     public float angle = 0;
%     public float turnSensitivity = 1.0F;
%     public float rudderLerpSpeed = 5.0F;
%     public Transform[] rudder;

%     public float keelEfficiency = 0.5F;
%     private Rigidbody boatRb;
%     private float targetAngle = 0;
%     private bool inputReceived = false; // Flag to check if any rudder input was received

%     [SerializeField] private float boatForwardSpeed = 0;

%     public float rotationScale = 10.0F;
%     private void Awake()
%     {
%         boatRb = GetComponentInParent<Rigidbody>();
%     }

%     void Update()
%     {
%         // Smoothly interpolate between the current angle and the target angle
%         angle = Mathf.Lerp(angle, targetAngle, rudderLerpSpeed * Time.deltaTime);

%         // Calculate the forward speed of the boat
%         boatForwardSpeed = Vector3.Dot(boatRb.velocity, transform.forward);

%         // Adjust the rudder angle based on boat's forward speed
%         // float adjustedAngle = angle * turnSensitivity * boatForwardSpeed;
%         float adjustedAngle = angle * turnSensitivity;

%         for (int i = 0; i < rudder.Length; i++)
%             rudder[i].localRotation = Quaternion.Euler(0, adjustedAngle, 0);

%         // Apply turning force based on rudder's angle and boat's speed
%         float turnForce = angle * boatForwardSpeed * rotationScale;
%         boatRb.AddTorque(transform.up * turnForce);

%         // if (boatForwardSpeed > 0.25 || boatForwardSpeed < -1)
%         // {
%             Vector3 lateralVelocity = Vector3.Dot(boatRb.velocity, transform.right) * transform.right;
%             boatRb.velocity -= lateralVelocity;
%         // }
        


%         // If no input is received, set target angle back to center
%         if (!inputReceived)
%         {
%             targetAngle = 0;
%         }

%         // Reset input flag for next frame
%         inputReceived = false;
%     }

%     public void RudderLeft()
%     {
%         targetAngle -= 1F;
%         targetAngle = Mathf.Clamp(targetAngle, -90F, 90F);
%         inputReceived = true;
%     }

%     public void RudderRight()
%     {
%         targetAngle += 1F;
%         targetAngle = Mathf.Clamp(targetAngle, -90F, 90F);
%         inputReceived = true;
%     }

%     public float GetBoatForwardSpeed()
%     {
%         return boatForwardSpeed;
%     }

%     public void SetRudderTargetAngle(float targetAngle)
%     {
%         this.targetAngle = targetAngle;
%     }
%     }
    
% \end{lstlisting}
% \subsection{Kite Model}\label{sec:kite}
% \begin{lstlisting}
%     using System.Collections;
%     using System.Collections.Generic;
%     using UnityEngine;
%     using UnityEngine.Rendering.HighDefinition;
    
%     public class KiteSim : MonoBehaviour
%     {
%         public Vector3 currentResultantForce;
%         public float forceMultiplier = 1.0f;
%         public WindSim windSim = null;
%         public float lineLength = 35.0f;
%         public ConfigurableJoint linePowerLeft, linePowerRight, lineSteeringLeft, lineSteeringRight;
%         public float liftCoefficient = 0.3f, dragCoefficient = 0.1f, _dragCoefficient = 0.1f;
%         public float barPower = 0.15f;
%         public SphereCollider kiteCollider;
%         public float debugScale = 1.0f;
%         public Rigidbody KiteRigidbody, BaseRigidbody;
%         public Transform windIndicator;
%         public Vector3 windAtKite;
%         private float time;
%         public int counter = 0;
%         public float score = 0.0f;
%         public Vector2 BarPositionAsControlInput;
%         private float stepsPerSecond = 0f;
%         public bool isResetting = false;
%         private int resetCounter = 0;
%         private bool isSteeringLeft = false, isSteeringRight = false;
%         public float steerSpeed = 0.1f, returnSpeed = 0.05f;
%         private Vector3 startPosition, boatStartPosition;
%         private Quaternion startRotation, boatStartRotation;
    
%         private void Start()
%         {
%             time = 0;
%             UpdateWindAtKite();
%             stepsPerSecond = 1 / Time.fixedDeltaTime;
%             InvokeRepeating("UpdateWindAtKite", 0, 5);
%             startPosition = KiteRigidbody.transform.position;
%             startRotation = KiteRigidbody.transform.rotation;
%             boatStartPosition = BaseRigidbody.transform.position;
%             boatStartRotation = BaseRigidbody.transform.rotation;
%         }
    
%         private void UpdateWindAtKite()
%         {
%             windAtKite = windSim.wind;
%         }
    
%         private void Update()
%         {
%             time += Time.deltaTime;
%         }
    
%         private float calculateAngleOfAttack(Vector3 apparentWind)
%         {
%             return 180 - Vector3.Angle(apparentWind, KiteRigidbody.transform.forward);
%         }
    
%         public void SteerLeft()
%         {
%             isSteeringLeft = true;
%             isSteeringRight = false;
%         }
    
%         public void SteerRight()
%         {
%             isSteeringRight = true;
%             isSteeringLeft = false;
%         }
    
%         public void StopSteering()
%         {
%             isSteeringLeft = false;
%             isSteeringRight = false;
%         }
    
%         private void UpdateBarPosition()
%         {
%             if (isSteeringLeft)
%             {
%                 BarPositionAsControlInput.x = Mathf.Max(BarPositionAsControlInput.x - steerSpeed, -1f);
%             }
%             else if (isSteeringRight)
%             {
%                 BarPositionAsControlInput.x = Mathf.Min(BarPositionAsControlInput.x + steerSpeed, 1f);
%             }
%             else
%             {
%                 BarPositionAsControlInput.x = Mathf.Lerp(BarPositionAsControlInput.x, 0, returnSpeed);
%             }
%         }
    
%         public void ResetKite()
%         {
%             KiteRigidbody.velocity = Vector3.zero;
%             KiteRigidbody.angularVelocity = Vector3.zero;
%             BaseRigidbody.velocity = Vector3.zero;
%             BaseRigidbody.angularVelocity = Vector3.zero;
%             Rigidbody keelRigidbody = BaseRigidbody.transform.GetChild(0).GetChild(0).GetComponent<Rigidbody>();
%             keelRigidbody.velocity = Vector3.zero;
%             keelRigidbody.angularVelocity = Vector3.zero;
%             StopSteering();
%             KiteRigidbody.transform.position = startPosition;
%             KiteRigidbody.transform.rotation = startRotation;
%             BaseRigidbody.transform.position = boatStartPosition;
%             BaseRigidbody.transform.rotation = boatStartRotation;
%             score = 0.0f;
%             counter = 0;
%         }
    
%         private void FixedUpdate()
%         {
%             counter++;
%             UpdateBarPosition();
%             Vector3 apparentWind = windAtKite - KiteRigidbody.velocity;
%             float angleOfAttack = calculateAngleOfAttack(apparentWind);
%             float adjustedLiftCoefficient = liftCoefficient * Mathf.Cos((angleOfAttack - 20f) * Mathf.Deg2Rad) * liftCoefficient;
%             float apparentWindSpeedSquared = apparentWind.sqrMagnitude;
%             Vector3 liftForce = Vector3.Cross(apparentWind, -KiteRigidbody.transform.right).normalized * apparentWindSpeedSquared / 2 * adjustedLiftCoefficient;
%             Vector3 dragForce = apparentWind.normalized * apparentWindSpeedSquared / 2 * _dragCoefficient;
%             currentResultantForce = liftForce + dragForce;
%             Vector3 forceOnBoat = currentResultantForce * forceMultiplier;
%             KiteRigidbody.AddForce(forceOnBoat);
%             float torqueMultiplier = 2f * KiteRigidbody.angularVelocity.magnitude + 10f * KiteRigidbody.velocity.magnitude + 1.8f;
%             KiteRigidbody.AddTorque(-BarPositionAsControlInput.x * torqueMultiplier * KiteRigidbody.transform.up);
%             KiteRigidbody.angularVelocity = Vector3.ClampMagnitude(KiteRigidbody.angularVelocity, 3f);
%             Vector3 localVelocity = KiteRigidbody.transform.InverseTransformDirection(KiteRigidbody.velocity);
%             localVelocity.x *= 0.9f;
%             KiteRigidbody.velocity = KiteRigidbody.transform.TransformDirection(localVelocity);
%             if (counter < stepsPerSecond * 5)
%             {
%                 score += 0.02f;
%             }
%             ResetForces();
%         }
    
%         private void ResetForces()
%         {
%             currentResultantForce = Vector3.zero;
%             currentResultantForce += linePowerLeft.currentForce;
%             currentResultantForce += linePowerRight.currentForce;
%             currentResultantForce += lineSteeringLeft.currentForce;
%             currentResultantForce += lineSteeringRight.currentForce;
%         }
    
%         private void OnDrawGizmos()
%         {
%             Vector3 normalizedResultantForce = currentResultantForce.normalized;
%             Gizmos.color = Color.white;
%             Gizmos.DrawLine(lineSteeringRight.transform.position, lineSteeringRight.transform.position + normalizedResultantForce * 35 * debugScale);
%         }
%     }
    
% \end{lstlisting}  


% \subsection{GJK Collision Detection}\label{sec:gjk}
% \begin{lstlisting}
%   using UnityEngine;
%   using System.Collections.Generic;
  
%   public class GJKCollisionDetection
%   { 
  
%       Vector3 ORIGIN = Vector3.zero;
%       public bool GJK(MeshFilter mf1, Vector3 mf1Pos, MeshFilter mf2, Vector3 mf2Pos)
%       {
%           Vector3[] vertices1 = mf1.mesh.vertices;
%           Vector3[] vertices2 = mf2.mesh.vertices;
%           Vector3 d = (mf2Pos - mf1Pos).normalized;
%           d.y = 0;
  
%           Vector3 simplexPoint = Support(vertices1, mf1.transform, vertices2, mf2.transform, d);
%           List<Vector3> simplex = new List<Vector3> { simplexPoint };
  
%           d = -simplexPoint;
  
%           while (true)
%           {
%               Vector3 A = Support(vertices1, mf1.transform, vertices2, mf2.transform, d);
%               A.y = 0;
%               if (Vector3.Dot(A, d) < 0)
%               {
%                   return false;
%               }
  
%               simplex.Add(A);
%               if (HandleSimplex(simplex, ref d))
%               {
%                   return true;
%               }
%           }
%       }
  
%       private Vector3 Support(Vector3[] vertices1, Transform t1, Vector3[] vertices2, Transform t2, Vector3 d)
%       {
%           Vector3 p1 = GetFarthestPointInDirection(vertices1, t1, d);
%           Vector3 p2 = GetFarthestPointInDirection(vertices2, t2, -d);
  
%           return p1 - p2;
%       }
  
%       private Vector3 GetFarthestPointInDirection(Vector3[] vertices, Transform transform, Vector3 d)
%       {
%           float highest = -Mathf.Infinity;
%           Vector3 supportVertex = Vector3.zero;
  
%           foreach (Vector3 vertex in vertices)
%           {
%               Vector3 vertexWorld = transform.TransformPoint(vertex);
%               float dot = Vector3.Dot(vertexWorld, d);
%               if (dot > highest)
%               {
%                   highest = dot;
%                   supportVertex = vertexWorld;
%               }
%           }
  
%           return supportVertex;
%       }
  
%       private bool HandleSimplex(List<Vector3> simplex, ref Vector3 d)
%       {
%           if (simplex.Count == 2)
%           {
%               return LineCase(simplex, ref d);
%           }
%           else
%           {
%               return TriangleCase(simplex, ref d);
%           }
%       }
  
%       private bool LineCase(List<Vector3> simplex, ref Vector3 d)
%       {
%           Vector3 B = simplex[0];
%           Vector3 A = simplex[1];
%           Vector3 AB = B - A;
%           Vector3 AO = ORIGIN - A;
%           Vector3 ABperp = TripleProd(AB, AO, AB);
%           d = ABperp;
%           return false;
%       }
  
%       private bool TriangleCase(List<Vector3> simplex, ref Vector3 d)
%       {
%           Vector3 C = simplex[0];
%           Vector3 B = simplex[1];
%           Vector3 A = simplex[2];
%           Vector3 AB = B - A;
%           Vector3 AC = C - A;
%           Vector3 AO = ORIGIN - A;
  
%           Vector3 ABperp = TripleProd(AC, AB, AB);
%           Vector3 ACperp = TripleProd(AB, AC, AC);
  
%           if (Vector3.Dot(ABperp, AO) > 0)
%           {
%               simplex.RemoveAt(0);
%               d = ABperp;
%               return false;
%           }
%           else if (Vector3.Dot(ACperp, AO) > 0)
%           {
%               simplex.RemoveAt(1);
%               d = ACperp;
%               return false;
%           }
  
%           return true;
%       }
  
%       private Vector3 TripleProd(Vector3 a, Vector3 b, Vector3 c)
%       {
%           Vector3 result = b * Vector3.Dot(c, a) - a * Vector3.Dot(c, b);
%           result.y = 0;
%           return result;
%       }
%   }

% \end{lstlisting}


\subsection{Grid Search}\label{gridsearch}
\begin{lstlisting}[language=Python]
import itertools
import os

# Define the ranges for each hyperparameter you want to vary, with fewer options
batch_size_options = [256, 512]
buffer_size_options = [2048, 4096]
learning_rate_options = [1.0e-4, 3.0e-4]
beta_options = [1.0e-4, 5.0e-4]
epsilon_options = [0.2, 0.3]
lambd_options = [0.95, 0.99]
num_epoch_options = [3, 4]
hidden_units_options = [128, 256]
num_layers_options = [4, 5]

# Create a product of all the hyperparameter options
grid_search = list(itertools.product(
    batch_size_options,
    buffer_size_options,
    learning_rate_options,
    beta_options,
    epsilon_options,
    lambd_options,
    num_epoch_options,
    hidden_units_options,
    num_layers_options
))

# If there are more than 100 configurations, randomly sample 100 from them
import random
if len(grid_search) > 100:
    grid_search = random.sample(grid_search, 100)

    # Create a directory for the config files if it doesn't exist
config_directory = "config_files"
os.makedirs(config_directory, exist_ok=True)

# Function to generate the config file content
def generate_config_content(batch_size, buffer_size, learning_rate, beta, epsilon, 
lambd, num_epoch, hidden_units, num_layers):

    return f"""behaviors:
  BoatAgent:
    trainer_type: ppo
    hyperparameters:
    batch_size: {batch_size}
    buffer_size: {buffer_size}
    learning_rate: {learning_rate}
    beta: {beta}
    epsilon: {epsilon}
    lambd: {lambd}
    num_epoch: {num_epoch}
    learning_rate_schedule: constant
    network_settings:
    normalize: false
    hidden_units: {hidden_units}
    num_layers: {num_layers}
    reward_signals:
    extrinsic:
    gamma: 0.99
    strength: 1.0
    curiosity:
    strength: 0.1
    gamma: 0.99
    learning_rate: {learning_rate}
    max_steps: 1000000
    time_horizon: 64
    summary_freq: 10000
    checkpoint_interval: 50000
    keep_checkpoints: 25
    """
    
    # Generate and save the config files
    for idx, config in enumerate(grid_search):
    batch_size, buffer_size, learning_rate, beta, epsilon, lambd, num_epoch, 
    hidden_units, num_layers = config

    config_content = generate_config_content(batch_size, buffer_size, 
    learning_rate, beta, epsilon, lambd, num_epoch, hidden_units, num_layers)

    config_filename = f"config_{idx+1:03d}.yaml"
    config_filepath = os.path.join(config_directory, config_filename)
    with open(config_filepath, 'w') as file:
    file.write(config_content)
    
    print(f"Generated {len(grid_search)} configuration files in the directory 
    '{config_directory}'.")
    
    
  \end{lstlisting}

  \subsection{HPC Shell Script}\label{sec:shell_script}
  \begin{lstlisting}[language=bash]
    #!/bin/bash
#SBATCH --account=COSC027924
#SBATCH --job-name=ai_kiteboat
#SBATCH --partition=cpu
#SBATCH --nodes=1
#SBATCH --ntasks-per-node=1
#SBATCH --cpus-per-task=14
#SBATCH --time=2-10:00:00
#SBATCH --mem=5000M


# 1. Create and activate a Python virtual environment.
python3 -m venv myenv
source myenv/bin/activate
pip install --upgrade pip
pip install pydrive

# 2. Install the ML Agents toolkit.
python -c "
try:
    import mlagents
    print('ml-agents already installed')
except ImportError:
    import os
    os.system('pip install mlagents==0.27.0')
    print('Installed ml-agents')
"

# 3. Download the config file and Unity executable from Google Drive.

# Direct download links converted from the shared URLs
yaml_link="https://drive.google.com/uc?export=download&id=..."
creds_link="https://drive.google.com/uc?export=download&id=..."
setting_link="https://drive.google.com/uc?export=download&id=..."

# navigate to the directory where the files will be downloaded ~/ML_PPO
mkdir ML_PPO
cd ~/ML_PPO

wget -O trainer_config.yaml $yaml_link
wget -O credentials.json $creds_link
wget -O settings.yaml $setting_link

fileid="..."
filename="build.zip"
query=`curl -c ./cookie.txt -s -L "https://drive.google.com/uc?export=download&id=
${fileid}" |\
     perl -nE'say/confirm=(\w+)/'`
curl -Lb ./cookie.txt "https://drive.google.com/uc?export=download&confirm=${query}&id=
{fileid}" -o ${filename}


# Unzip the Unity build folder
unzip  -o build.zip 

# 4. Run the Unity executable and train the ML Agents.
env_name="./HeadlessBuilds/build.x86_64"
trainer_config_file="./trainer_config.yaml"
run_identifier="test_1"

chmod +x $env_name 
chmod +x $trainer_config_file
chmod +x HeadlessBuilds

mkdir -p results
python -c "

from mlagents_envs.environment import UnityEnvironment
env = UnityEnvironment(file_name='$env_name', no_graphics=True)

env.reset()

behavior_name = list(env.behavior_specs)[0]
spec = env.behavior_specs[behavior_name]


max_steps = 1000  # Adjust this value based on your desired max steps
current_steps = 0

while current_steps < max_steps:
    env.reset()
    decision_steps, terminal_steps = env.get_steps(behavior_name)
    tracked_agent = -1 # -1 indicates not yet tracking
    done = False # For the tracked_agent
    episode_rewards = 0 # For the tracked_agent
    while not done:
        # Track the first agent we see if not tracking
        if tracked_agent == -1 and len(decision_steps) >= 1:
            tracked_agent = decision_steps.agent_id[0]

        # Generate an action for all agents
        action = spec.action_spec.random_action(len(decision_steps))

        # Set the actions
        env.set_actions(behavior_name, action)

        # Move the simulation forward
        env.step()
        current_steps += 1

        # Get the new simulation results
        decision_steps, terminal_steps = env.get_steps(behavior_name)
        if tracked_agent in decision_steps:  # The agent requested a decision
            episode_rewards += decision_steps[tracked_agent].reward
        if tracked_agent in terminal_steps:  # The agent terminated its episode
            episode_rewards += terminal_steps[tracked_agent].reward
            done = True
    print('Total rewards until now: ' + str(episode_rewards))


env.close()
print('Closed environment')
"
# ls -la

# 5. Zip the results file and upload it to Google Drive.
results_folder="results"
zip_file="${results_folder}.zip"
zip -r $zip_file $results_folder


python -c "
from pydrive.auth import GoogleAuth
from pydrive.drive import GoogleDrive

# Authentication
gauth = GoogleAuth()
# gauth.LoadCredentialsFile('credentials.json')
gauth.CommandLineAuth()

if gauth.credentials is None:
    print('Missing credentials.json')
    exit(1)

drive = GoogleDrive(gauth)

# Upload the zipped results file
file_drive = drive.CreateFile({'title': '$zip_file'})
file_drive.SetContentFile('$zip_file')
file_drive.Upload()
# print('Uploaded ' + zip_file + ' to Google Drive')
"

\end{lstlisting}
  
\subsection{Extract Results}\label{sec:extract_results}
\begin{lstlisting}[language=Python]
  import os
  import pandas as pd
  from tensorboard.backend.event_processing import event_accumulator
  
  # Directory where the ML-Agents results are stored
  results_directory = 'results'
  
  # Function to extract scalar data from a single event file
  def extract_scalar_data_from_event_file(event_file_path):
      # Create an accumulator to collect the data from the event file
      ea = event_accumulator.EventAccumulator(event_file_path)
      ea.Reload()
      print(f'Extracting data from {event_file_path}')
      
      # Extract scalar data
      scalar_data = {}
      for tag in ea.Tags()['scalars']:
          events = ea.Scalars(tag)
          scalar_data[tag] = [(e.step, e.value) for e in events]
      
      return scalar_data
  
  # Function to convert the extracted data into a pandas DataFrame and format it for readability
  def scalar_data_to_dataframe(scalar_data):
      dataframes = {}
      for tag, values in scalar_data.items():
          df = pd.DataFrame(values, columns=['Step', 'Value'])
          df['Tag'] = tag
          # Format the 'Value' column to round to 4 decimal places for readability
          df['Value'] = df['Value'].round(4)
          dataframes[tag] = df
  
      if dataframes:
          # Concatenate all dataframes and sort by 'Tag' and 'Step' for better readability
          return pd.concat(dataframes.values(), ignore_index=True).sort_values(by=['Tag', 'Step'])
      else:
          return pd.DataFrame(columns=['Step', 'Value', 'Tag'])
  
  # Find all event files in the results directory
  event_files = [os.path.join(dirpath, f)
                 for dirpath, dirnames, files in os.walk(results_directory)
                 for f in files if f.startswith('events.out.tfevents')]
  
  # Process each event file and save the results to a CSV file
  for event_file in event_files:
      scalar_data = extract_scalar_data_from_event_file(event_file)
      
      if scalar_data:
          df = scalar_data_to_dataframe(scalar_data)
          # Determine the directory for the current event file
          run_directory = os.path.dirname(os.path.dirname(event_file))
          # Save the DataFrame to a CSV file in the run directory
          csv_file_path = os.path.join(run_directory, 'results.csv')
          df.to_csv(csv_file_path, index=False)
          print(f'Results saved to {csv_file_path}')
      else:
          print(f'No scalar data found in {event_file}')

\end{lstlisting}

\subsection{Combine Results}\label{sec:combine_results}
\begin{lstlisting}
  import pandas as pd
import os
import yaml

results_dir = 'results_100x2'
combined_data = []

# Specify the nested keys for the varying parameters you want to extract
varying_params = {
    'batch_size': ['behaviors', 'BoatAgent', 'hyperparameters', 'batch_size'],
    'buffer_size': ['behaviors', 'BoatAgent', 'hyperparameters', 'buffer_size'],
    'learning_rate': ['behaviors', 'BoatAgent', 'hyperparameters', 'learning_rate'],
    'beta': ['behaviors', 'BoatAgent', 'hyperparameters', 'beta'],
    'epsilon': ['behaviors', 'BoatAgent', 'hyperparameters', 'epsilon'],
    'lambd': ['behaviors', 'BoatAgent', 'hyperparameters', 'lambd'],
    'num_epoch': ['behaviors', 'BoatAgent', 'hyperparameters', 'num_epoch'],
    'hidden_units': ['behaviors', 'BoatAgent', 'network_settings', 'hidden_units'],
    'num_layers': ['behaviors', 'BoatAgent', 'network_settings', 'num_layers'],
    'curiosity_strength': ['behaviors', 'BoatAgent', 'reward_signals', 'curiosity', 'strength'],
    'curiosity_learning_rate': ['behaviors', 'BoatAgent', 'reward_signals', 'curiosity', 'learning_rate']
}

def get_nested_value(dct, keys):
    for key in keys:
        dct = dct.get(key, {})
    return dct if isinstance(dct, (int, float)) else None

# Loop through each subdirectory in the results directory
for model_dir in os.listdir(results_dir):
    model_path = os.path.join(results_dir, model_dir)
    if os.path.isdir(model_path):
        config_path = os.path.join(model_path, 'configuration.yaml')
        results_path = os.path.join(model_path, 'results.csv')
        
        with open(config_path, 'r') as yaml_file:
            config = yaml.safe_load(yaml_file)

        # Extract only the varying parameters using the nested keys
        config_values = {param: get_nested_value(config, keys) for param, keys in varying_params.items()}

        # Check if the results.csv exists before trying to read it
        if os.path.exists(results_path):
            df_results = pd.read_csv(results_path)
            df_episode_length = df_results[df_results['Tag'] == 'Environment/Episode Length']
            best_episode_length = df_episode_length['Value'].max()
            df_av_loss = df_results[df_results['Tag'] == 'Losses/Value Loss']
            av_loss = df_av_loss['Value'].mean()
            df_av_curiosity = df_results[df_results['Tag'] == 'Policy/Curiosity']
            av_curiosity = df_av_curiosity['Value'].mean()
            df_av_reward = df_results[df_results['Tag'] == 'Policy/Extrinsic Reward']
            av_reward = df_av_reward['Value'].mean()

        else:
            best_episode_length = float('inf')  # Use infinity to represent missing or unobtainable lengths

        combined_data.append({
            'Model': model_dir,
            **config_values,
            'Best Episode Length': best_episode_length,
            'Av Loss': av_loss,
            'Av Curiosity': av_curiosity,
            'Av Reward': av_reward
        })

# Convert the combined data to a DataFrame
df_combined = pd.DataFrame(combined_data)

# Convert 'Best Episode Length' to numeric, setting errors='coerce' to handle non-numeric entries
df_combined['Best Episode Length'] = pd.to_numeric(df_combined['Best Episode Length'], errors='coerce')

# Drop rows where 'Best Episode Length' could not be converted to a number
df_combined.dropna(subset=['Best Episode Length'], inplace=True)

# Sort by 'Best Episode Length'
df_combined.sort_values('Best Episode Length', ascending=False, inplace=True)

# Save the sorted DataFrame to a CSV file
output_csv_path = os.path.join(results_dir, 'combined_results.csv')
df_combined.to_csv(output_csv_path, index=False)

print(f'Combined CSV file created at: {output_csv_path}')

\end{lstlisting}


\subsection{Analysis}\label{sec:analysis}

\begin{lstlisting}
  import matplotlib.pyplot as plt
import seaborn as sns
from sklearn.ensemble import RandomForestRegressor
from sklearn.model_selection import train_test_split
from sklearn.metrics import mean_squared_error
import pandas as pd

# Load the dataset
data = pd.read_csv('combined_results.csv')

# Define the features (config parameters) and the target (Best Episode Length)
features = data.drop(columns=['Model', 'Best Episode Length', 'Av Loss', 'Av Curiosity', 'Av Reward'])
target = data['Best Episode Length']

# Split the data into training and testing sets
X_train, X_test, y_train, y_test = train_test_split(features, target, test_size=0.3, random_state=42)

# Initialize the Random Forest Regressor
rf = RandomForestRegressor(n_estimators=100, random_state=42)

# Train the model
rf.fit(X_train, y_train)

# Predict on the test set
y_pred = rf.predict(X_test)

# Calculate the feature importances
feature_importances = rf.feature_importances_

# Calculate the Mean Squared Error on the test set
mse = mean_squared_error(y_test, y_pred)

# Visualize the feature importances
sns.barplot(x=feature_importances, y=features.columns)
plt.xlabel('Importance')
plt.ylabel('Features')
# plt.savefig('feature_importances.png', dpi=300)
plt.show()

# Scatter plot matrix for the most correlated variables
most_correlated = data[['Best Episode Length', 'num_epoch', 'buffer_size', 'batch_size', 'epsilon']]
sns.pairplot(most_correlated)
plt.suptitle('Scatter plot matrix for the most correlated variables', y=1.02)  # y=1.02 to adjust the title position
plt.savefig('scatter_plot_matrix.png', dpi=300)

mse, feature_importances
print(mse)

statistical_summary = data.describe()

# Computing the correlation matrix to see the relationship between parameters and performance metric

correlation_matrix = data.corr(method='spearman')

(statistical_summary, correlation_matrix['Best Episode Length'].sort_values(ascending=False))
print(correlation_matrix['Best Episode Length'].sort_values(ascending=False))
\end{lstlisting}

% \subsection{Plot Results}\label{sec:plot_results}

% \section{Results}

% \subsection{Environment/Cumulative Reward}\label{sec:env_cum_reward}
% insert code here





\